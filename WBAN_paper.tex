%
% latex-sample.tex
%
% This LaTeX source file provides a template for a typical research paper.
%

%
% Use the standard article template.
%
\documentclass[twocolumn,10pt]{article}

\usepackage{titling}


\usepackage{float}
                   % collegamenti ipertestuali

\usepackage[babel]{csquotes}

\usepackage{lastpage}
\usepackage{fancyhdr}
\usepackage{booktabs,tabularx}
\usepackage{makeidx}
\usepackage{fixltx2e}
%\usepackage{hyperref}


% The geometry package allows for easy page formatting.
\usepackage{geometry}
\geometry{letterpaper}

% Load up special logo commands.
\usepackage{doc}

% Package for formatting URLs.
\usepackage{url}
\usepackage[utf8]{inputenc}
\usepackage{caption}

% Packages and definitions for graphics files.
\usepackage{graphicx}
\usepackage{epstopdf}
\usepackage{stfloats}
\usepackage{adjustbox} 
\usepackage[colorlinks, linkcolor = blue, urlcolor = blue]{hyperref}
\DeclareGraphicsRule{.tif}{png}{.png}{`convert #1 `dirname #1`/`basename #1 .tif`.png}

\usepackage[final]{pdfpages}
\usepackage{shellesc}

\usepackage{tabularx,pbox}
\usepackage{here}
\usepackage[titletoc,title]{appendix}


%
% Set the title, author, and date.
%
\title{\LARGE \textbf{An Overview on Wireless Body Area Networks}}
\author{Rudy Berton \and Vassilikì Menarin }
\date{\small {Email: rudy.bertom@studenti.unipd.it vassiliki.menarin@studenti.unipd.it }}

\def\code#1{\texttt{#1}}
\def \ped#1{\textsubscript{#1}}
\def \sup#1{\textsuperscript{#1}}

%
% The document proper.
%
\begin{document}
	
	% Add the title section.

	\maketitle
	
	\textbf{\textit{Abstarct} - 
		 CONTINUA}

	
	\section{Introduction}	
	Spunti: Utilizzo delle ban, perche- puo- avere senso studiarle, dove trovano la loro maggiore applicazine
	
	\section{}
	Spunti:
			
	\subsection{Implementations}
	Spunti: discussione sui possibili protocolli che possono essere usati e loro breve storia? Una sottosezione per ogni protocollo?
	
	Esempio figura:		
	\begin{figure}[!h]
		\centering
		\includegraphics[width=0.8\linewidth]{img/BG01.png}
		\caption{Extensive form of the Bayesan Game considered}				
		\label{fig:BG01}
	\end{figure}

\section{Comparison}
Spunti> possibile confronto tra i protocolli presentati sopra?

\section{Implementations}
Spunti: Usi e applicazioni delle BAN?

\section{Major research issues/ Challenges}
Se c'e' da dire

\section{Conclusion}
Conclusione

\clearpage

% Generate the bibliography.
	\begin{thebibliography}{9}	
	\bibitem{1} 
	Esempio bib
	
	\bibitem{2} 
	Inserisci
	
	\end{thebibliography}
	%\bibliography{latex-sample}
	%\bibliographystyle{unsrt}
	
\end{document}
